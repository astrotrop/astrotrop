





\subsection{VO services}

\subsection{The registry}

The VO Registry provides the first layer of data discovery available in the
virtual observatory. The individual registry services deployed at participating
institutes work together to provide a shared repository for describing datasets,
data access services and data processing services in a standard way.

The IVOA Registry Interfaces standard
\endnote{http://www.ivoa.net/documents/RegistryInterface/}
defines the web service interfaces that support interactions between
applications and registries as well as between the registries themselves.

The high level structure of the registry content is defined by a
set of IVOA standards, including a standard format for IVOA Identifiers
\endnote{http://www.ivoa.net/documents/latest/IDs.html}
and the basic Resource Metadata
\endnote{http://www.ivoa.net/Documents/latest/RM.html}

The details of the registry metadata are covered by a set of technical
standards defining the detailed XML schemas for resource metadata.
\begin{itemize}
  \item VOResource
  \endnote{http://www.ivoa.net/documents/latest/VOResource.html}
  \item VODataService
  \endnote{http://www.ivoa.net/documents/VODataService/} 
  \item Simple Data Access Services
  \endnote{http://www.ivoa.net/documents/SimpleDALRegExt/20131005/}
\end{itemize}

\subsection{Service registration}

VOSI \ldots and stuff \ldots

\subsection{Service metadata}

registry metadata queries \ldots and stuff \ldots

\subsection{Service footprint}

registry footprint queries \ldots and stuff \ldots

HEALPix Multi-Order Coverage Map (MOC)
\endnote{http://www.ivoa.net/documents/MOC/}

\subsection{Data access services}

The VO DataAccess services can be categorised as two types of services.

A set of type specific data discovery servces which are designed to provide
simple service interfaces for discovering and acessing data of a specific type.

\begin{itemize}
  \item Simple Cone Search (SCS)
  \endnote{http://www.ivoa.net/documents/latest/ConeSearch.html}
  \item Simple Image Access (SIA)
  \endnote{http://www.ivoa.net/documents/SIA/}
  \item Simple Spectral Access (SSA)
  \endnote{http://www.ivoa.net/documents/SSA/}
  \item Simple Line Access (SLA)
  \endnote{http://www.ivoa.net/documents/SLAP/}
\end{itemize}

A tabular data access services for querying tabluar data using a common query
derrived from SQL.

\begin{itemize}
  \item Table Access Protocol (TAP)
  \endnote{http://www.ivoa.net/Documents/TAP/}
  \item Astronomy Data Query Language (ADQL)
  \endnote{http://www.ivoa.net/Documents/latest/ADQL.html}
\end{itemize}

\subsubsection{Simple Image Access}

The Simple Image Access (SIA) protocol provides 
\begin{quote}
parameter based discovery of images and datacubes, querying the service(s) with
a few well known kinds of queries that cover greater than 95\% of use, and
getting back easily parsed summary metadata about each available data product
\end{quote}

The Simple Image Access (SIA) data discovery service provides support for
the following use cases:

\begin{itemize}
  \item find data that includes specified coordinates (e.g. for some object) 
  \item find data in the circle with coordinate centre and radius 
  \item find data in a range of longitude and latitude 
  \item find data within a specified simple  polygon (one region, no holes, less
  than half the sphere)
  \item find data containing a specified energy (e.g. wavelength) or in a
  specified range of energy values
  \item find data obtained at a specified time (e.g. including a time instant)
  or during a specified range of times
  \item find data obtained with specified polarization (Stokes) states 
  \item find data within a specified range of spatial resolution 
  \item find data within a specified range of field-of-view 
  \item find data within a range of exposure (integration) time 
\end{itemize}

The response from a successful SIA data discovery query is a VOTable containing
instances of the ObsCore \endnote{http://www.ivoa.net/documents/ObsCore/} data
model.

Each row in the results corresponds to a data product that matches the search
criteria and includes details of how to access the data products or how
to request additional matadata.

\subsubsection{Simple Spectral Access}

The Simple Spectral Access (SSA) protocol is similar to the Simple Image Access (SIA) protocol.

The primary differences are the type of data searched for, and the set of query
parameters.

\begin{quote}
\ldots discover and access one dimensional spectra
\ldots based on a general data model capable of describing most tabular
spectrophotometric data, including time series and spectral energy distributions
(SEDs) as well as 1-D spectra
\end{quote}

\subsubsection{Simple Line Access}

The Simple Line Access (SLA) protocol is similar to the Simple Image Access
(SIA) protocol.

The primary differences are the type of data searched for, and the set of query
parameters.

\begin{quote}
\ldots retrieving spectral lines coming from various Spectral Line Data
Collections
\ldots either observed or theoretical and will be typically used to identify
emission or absorption features in astronomical spectra.
\ldots makes use of the Simple Spectral Line Data Model
(SSLDM) \endnote{http://www.ivoa.net/documents/SSLDM/} to characterize spectral
lines through the use of uTypes
\endnote{http://www.ivoa.net/documents/Notes/UTypesUsage/index.html}
\end{quote}
  
\subsubsection{Table Access Protocol}

Table Access Protocol (TAP) is a generic protocol for accessing general table
data, including astronomical catalogs as well as general database tables, with
support for both synchronous and asynchronous queries.

Special support is provided for spatially indexed queries using the spatial
extensions in ADQL.

A multi-position query capability permits queries against an arbitrarily large
list of astronomical targets, providing a simple spatial cross-matching
capability.

Deploying the same standard interface and query language across multiple sites
means that cross-matching queries are possible by orchestrating a distributed
query across multiple TAP services.

